%Paquetes y formato del documento. TODO: dividir esto en dos.
\usepackage[utf8]{inputenc}
\usepackage{multicol}

%\usepackage{apacite} por si quieres citar en formato APA.

\usepackage[usenames,dvipsnames, svgnames]{xcolor}
\definecolor{nuevo}{RGB}{47, 124, 247}
\newcommand{\TODO}[1]{\textcolor{purple}{#1}}



%Imágenes ---------------------------------
\usepackage{graphicx}
\graphicspath{{./imagenes}}
\usepackage{subfigure}
\usepackage{sidecap} %para poder usar el ambiente SCfigure



\usepackage{tabularx,ragged2e,booktabs,caption}
\usepackage{titlesec}
\usepackage[bookmarks,breaklinks,colorlinks=true,allcolors=blue]{hyperref}
\usepackage{listings}
\usepackage{inconsolata}
\usepackage{float}
\usepackage{eso-pic}
\usepackage[framemethod=tikz]{mdframed}


\usepackage{thmbox} %https://tex.stackexchange.com/questions/513107/what-is-this-fancy-theorem-environment


%Math packages
\usepackage{amsmath}
\usepackage{amssymb}




\usepackage{parskip}
\usepackage[official]{eurosym}
\setlength {\marginparwidth }{1cm} 
%\usepackage{todonotes}
\usepackage{csquotes}

\usepackage{rotating}
\usepackage{lmodern}
\usepackage{setspace}
\usepackage[most]{tcolorbox}





%--------------- Dimensiones del texto y notas laterales.
\usepackage{geometry}
\usepackage{sidenotes} % put annotations, tables and figures in the margin

\captionsetup{font=footnotesize, skip=4pt}
\usepackage{morefloats}
\usepackage{marginfix}
\geometry{
paperwidth=210mm,
paperheight=297mm,
left=72pt,
top=42pt,
textwidth=360pt,
marginparsep=20pt,
marginparwidth=110pt,
textheight=650pt,
footskip=40pt,
}


\DeclareCaptionStyle{sidecaption}[]{labelfont={sf,bf,footnotesize}, font=footnotesize, justification=justified, singlelinecheck=off}

\renewcommand{\normalsize}{\fontsize{10pt}{13pt}\selectfont}%
\renewcommand{\footnotesize}{\fontsize{8pt}{10pt}\selectfont}%
% fullwidth environment, text across textwidth+marginparsep+marginparwidth
\newlength{\overhang}
\setlength{\overhang}{\marginparwidth}
\addtolength{\overhang}{\marginparsep}
%
\newenvironment{fullwidth}
  {\ifthenelse{\boolean{@twoside}}%
     {\begin{adjustwidth*}{}{-\overhang}}%
     {\begin{adjustwidth}{}{-\overhang}}%
  }%
  {\ifthenelse{\boolean{@twoside}}%
    {\end{adjustwidth*}}%
    {\end{adjustwidth}}%
  }


%------------------------------------------
\let\newemptytheorem\newtheorem
\usepackage{thmbox}

\newemptytheorem{ejercicio}{Ejercicio}
%TODO así como en la tesis, definir environments de teorema en el que el marco no se aplique.

			%% Secciones:
			\newtheorem[M]{teo}{Teorema}[section]
			\newtheorem[M]{lema}[teo]{Lema}
			\newtheorem[M]{prop}[teo]{Proposición}
			\newtheorem[M]{obs}[teo]{Observación}
			\newtheorem[M]{dem}[teo]{Demostración}
			\newtheorem[M]{cor}[teo]{Corolario}
			\newtheorem[M]{defi}[teo]{Definición}
			\newtheorem[M]{ej}[teo]{Ejemplo}
			\newtheorem[M]{sol}{\bf Solución}
			\newtheorem[M]{nota}{\bf Nota}

%----- Símbolos matemáticos

			\newcommand*{\IR}{\mathbb{R}}
			\newcommand*{\QEDA}{\null\nobreak\hfill\ensuremath{\blacksquare}}% Final de demostración
			\newcommand*{\final}{\null\nobreak\hfill\ensuremath{\diamond}}
%Símbolo (un diamante) usado para marcar el final 
%de una definición o ejemplo.
			\newcommand*{\QEDB}{\null\nobreak\hfill\ensuremath{\square}}%
			\newcommand{\limite}[2]{\lim\limits_{#1}{#2}} %límites
			\newcommand{\suma}[3]{\sum\limits_{#1}^{#2}#3} %Sumas ys series
			\newcommand{\union}[3]{\bigcup\limits_{#1}^{#2}{#3}} %uniones
			\newcommand{\producto}[3]{\prod_{#1}^{#2}{#3}} %productos
			\newcommand{\inter}[3]{\bigcap\limits_{#1}^{#2}{#3}} %intersecc.
			\newcommand{\alg}{${{\sigma}}-$álgebra }
			\newcommand{\Om}{\Omega}
			\newcommand{\limitsup}[1]{\inter{m=1}{\infty}{\left(\union{n=m}{\infty}{#1}\right)}} %Límite superior
			\newcommand{\limitinf}[1]{\union{m=1}{\infty}{\left(\inter{n=m}{\infty}{#1}\right)}} %Límite inferior
			\newcommand\myeq[2]{\stackrel{\mathclap{\normalfont\mbox{#1}}}{#2}}
			%para escribir sobre símbolos matemátcos			
			
			
			\newcommand{\var}[1]{\sigma_{X}^{2}}
			\newcommand{\desv}[1]{\sigma_{X}}
			\newcommand{\mean}[1]{\mu_{X}}


		\newcommand{\aplica}[5]{\text{${\begin{array}{crcl} #1: & #2 & \longrightarrow & #3 \\ \, & #4 & \longmapsto & #5 \end{array}}$}} %Notación para la definición de una aplicación.

%-------------------------------------------

%\DeclareCaptionStyle{marginfigure}[]{labelfont={sf,bf,footnotesize}, font=footnotesize, justification=justified}


%Bilbiografía
\usepackage[backend=bibtex,style=alphabetic]{biblatex}
\addbibresource{fuentes.bib}